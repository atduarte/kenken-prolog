\documentclass[runningheads,a4paper]{llncs}

\usepackage{amssymb}
\setcounter{tocdepth}{3}
\usepackage{graphicx}
\usepackage[portuguese]{babel}
\usepackage[utf8]{inputenc}
\usepackage{indentfirst}
\usepackage{graphicx}
\usepackage{verbatim}

\usepackage[T1]{fontenc}
\usepackage{listings}
\usepackage{xcolor}
\usepackage[hidelinks]{hyperref}

\usepackage{url}
\urldef{\mailsa}\path|ei11044@fe.up.pt, ei11126@fe.up.pt|
\newcommand{\keywords}[1]{\par\addvspace\baselineskip
\noindent\keywordname\enspace\ignorespaces#1}

\usepackage{inconsolata}
\lstset{
    language=Prolog, %% Troque para PHP, C, Java, etc... bash é o padrão
    basicstyle=\ttfamily\small,
    numberstyle=\footnotesize,
    numbers=left,
    backgroundcolor=\color{gray!10},
    frame=single,
    tabsize=2,
    rulecolor=\color{black!30},
    escapeinside={\%*}{*)},
    breaklines=true,
    breakatwhitespace=true,
    framexbottommargin=2pt,
    extendedchars=false,
    inputencoding=utf8
}

\begin{document}

\mainmatter

\title{KenKen - Programação em Lógica\\Resolução de Problema de decisão/optimização\\
usando Restrições}

\titlerunning{KenKen - Resolução de Problema de Decisão}

\author{André Duarte e João Santos}

\authorrunning{KenKen - Resolução de Problema de Decisão}

\institute{Faculdade de Engenharia da Universidade do Porto,\\
Rua Roberto Frias, sn, 4200-465 Porto, Portugal\\
FEUP-PLOG, Turma 3MIEIC03, Grupo 41\\
\mailsa\\
\url{http://www.fe.up.pt}}


\toctitle{KenKen - Programação em Lógica}
\tocauthor{Resolucao de Problema de Decisão}
\maketitle

\begin{abstract}
KenKen é um puzzle baseado numa grelha que usa operações aritméticas básicas - soma, subtração, multiplicação e divisão - para criar desafios lógicos. A grelha de um jogo KenKen é constituída por áreas chamadas \textit{cages} às quais é atribuído um resultado e uma operação aritmética. O jogador tem que preencher as \textit{cages} de forma a que os números nela contidos, quando associados à operação especificada, dêem os resultados esperados, garantido que em cada linha ou coluna o mesmo número não seja repetido.
O objectivo deste trabalho é criar um algoritmo que crie e subsequentemente resolva grelhas kenken recorrendo à linguagem CLP utilizando o ambiente SICSTUS.
\keywords{KenKen, Puzzle, Restrições, Domínios, Prolog, CLP}
\end{abstract}

\section{Introdução}

Com este trabalho pretende-se obter conhecimentos de boas práticas de programação em lógica recorrendo a restrições e domínios para resolver problemas de decisão. Foi também uma oportunidade para experimentarmos funções nativas do SICSTUS das quais não tínhamos conhecimento até agora.
O objectivo deste trabalho é gerar e resolver um puzzle KenKen. O programa por nós desenvolvido gera um tabuleiro de tamanho variável NxN em que N pode ser qualquer inteiro de 3 a 9. O Tabuleiro é depois gerado pelo programa de acordo com as especificações fornecidas e posteriormente resolvido. As condicionantes serão melhor explicadas na secção 2 deste documento. No final, o programa mostra o problema gerado e a sua resolução.

\newpage

\section{Descrição do Problema}
O nosso trabalho está dividido em dois módulos principais, que foram desenvolvidos separadamente e depois chamados em conjunto para resolver o problema geral.

%************************************************************************************************
%************************************************************************************************

\subsection{Criação aleatória de tabuleiros KenKen}

 O tabuleiro é quadrado e está dividido em várias regiões, denominadas \textit{cages}, com formas variadas.  Cada linha deve conter todos os números de 1 a N (em que N é o lado do tabuleiro) exactamente uma vez. O mesmo deve ser garantido para cada coluna. A cada cage está atribuída uma operação aritmética e um resultado.

 Para a criação de um tabuleiro aleatório tivemos que lidar com uma situação não prevista na organização inicial do projecto: o Prolog responde sempre da mesma forma ao mesmo problema de restrições se as únicas condicionantes forem os predicados domain/3 e all\_different/1. Deste modo tivemos que arranjar uma maneira de fornecer números aleatórios sem quebrar as regras do tabuleiro. A nossa solução foi preencher a diagonal do tabuleiro com números aleatórios e desta forma garantimos que preenchendo o resto do tabuleiro recorrendo a restrições representativas das propriedades do tabuleiro obtemos sempre tabuleiros diferentes.

Para gerar as várias \textit{cages} tivemos que garantir que as operações de subtração e divisão eram apenas atribuídas a \textit{cages} com 2 elementos. No processo de criação da própria cage usamos um algoritmo semelhante ao de Prim. Começamos num elemento aleatório que não faça parte de nenhuma cage e a partir de aí construimos um caminho de tamanho aleatório e vamos acrescentando casas adjacentes aleatórias até chegar ao tamanho definido ou até não haverem mais possibilidades de caminho. É criada então uma grelha vazia e esta é então associada às cages para ser resolvida pelo programa, que depois devolve a nova grelha resolvida.


\subsection{Resolução de um qualquer tabuleiro KenKen}
Para resolver o problema tivemos apenas que seguir as restrições acima impostas e adicionar para cada \textit{cage} uma restrição relativa aos seus membros, operação e resultado, tendo em conta o número de casas de cada \textit{cage}.

\section{Ficheiros de Dados}

Apesar de não ser necessário, visto que geramos problemas aleatórios, definimos alguns problemas exemplo para poderem ser testados repetidamente.

Executando \textit{kenken(X)}, sendo X um número de 3 a 8 inclusive, o programa resolve estes problemas exemplo. Para gerar problemas aleatórios, deverá ser executado \textit{kenken(0, X)}, onde X é o tamanho do lado do tabuleiro.


Tanto a geração de problemas como os problemas exemplo encontram-se no ficheiro problems.pl.

\section{Variáveis de Decisão}

Cada elemento do Tabuleiro é uma variável de decisão e, visto que usamos uma lista de listas para representarmos o Tabuleiro, para fazermos o labeling usamos o append da seguinte forma:

\begin{lstlisting}
% Build unified list
append(Board, UniBoard),
...
% Label Variables
labeling([], UniBoard).
\end{lstlisting}

\section{Restrições}

No nosso problema todas as restrições são rígidas, visto que as regras são todas inflexíveis.

\subsection{Resolução do problema}

Para resolver o jogo, restringimos todos os elementos de cada linha e cada coluna a serem diferentes:

\begin{lstlisting}
% SETUP each row different
setupDifferent(Board),

% SETUP each column different
transpose(Board, TBoard),
setupDifferent(TBoard),
\end{lstlisting}

\begin{lstlisting}
setupDifferent([]).
setupDifferent([H|T]):-
   all_distinct(H),
   setupDifferent(T).
\end{lstlisting}

Também restringimos a multiplicação/soma/subtração/divisão dos elementos de cada \textit{cage} a ser igual ao resultado esperado:

\begin{lstlisting}
% SETUP groups
( foreach(P,Problem),
  param(Board)
  do
   setupGroup(Board, P)
),
\end{lstlisting}

Grupos unitários:

\begin{lstlisting}
setupGroup(Board, [Operation, Spaces, Result]) :-
    length(Spaces, Length),
    Length =:= 1,
    [[Row,Column]] = Spaces,
    getElement(Board, Row, Column, Position),
    Position is Result.
\end{lstlisting}

Grupos de tamanho 2:

\begin{lstlisting}
setupGroup(Board, [Operation, Spaces, Result]) :-
    length(Spaces, Length),
    Length =:= 2,
    [[Row1,Column1],[Row2,Column2]] = Spaces,
    getElement(Board, Row1, Column1, Position1),
    getElement(Board, Row2, Column2, Position2),
    checkGroup(Op, Position1, Position2, Result).
\end{lstlisting}

\begin{lstlisting}
checkGroup(1, Position1, Position2, Result) :- Position1 + Position2 #= Result.
checkGroup(2, Position1, Position2, Result) :- Position1 * Position2 #= Result.
checkGroup(3, Position1, Position2, Result) :- Position1 - Position2 #= Result.
checkGroup(4, Position1, Position2, Result) :- Position2 * Result #= Position1.
\end{lstlisting}

Grupos de tamanho maior que 2:

\begin{lstlisting}
setupGroup(Board, [Operation, Spaces, Result]) :-
    length(Spaces, Length),
    Length > 2,
    (foreach([Row,Column],Spaces),
     foreach(Value, UniSpaces),
     param(Board)
     do
         getElement(Board, Row, Column, Value)
    ),
    checkGroup(Operation, UniSpaces, Result).
\end{lstlisting}

\begin{lstlisting}
checkGroup(1, Positions, Result) :- sum(Positions,#=,Result).
checkGroup(2, Positions, Result) :- prod(Positions,Result).
\end{lstlisting}

\begin{lstlisting}
prod(List, Result) :-
        (foreach(L, List),
         fromto(1, In, Out, Result) do
             Out #= In * L
        ).
\end{lstlisting}

\subsection{Criação do problema}

As restrições usadas na fase de criação do tabuleiro base (sem \textit{cages}) são idênticas no que diz respeito aos elementos serem diferentes em cada linha e coluna. No entanto, de modo a garantir a aleatoriedade, colocamos elementos aleatórios na diagonal antes do \textit{labeling}.


\section{Estratégia de Pesquisa}

Recorrendo ao método experimental verificamos que para o nosso problema ao especificar apenas a opção bissect  obtemos os resultados mais rápidos e com menor utilização de recursos.

\section{Visualização do Problema}

O problema é representado \textit{cage} a \textit{cage}, sendo cada \textit{cage} uma lista com o operador, lista de casas e resultado.

\begin{lstlisting}
Problem:
[+,[[1,1],[1,2],[1,3]],7]
[+,[[1,4],[2,4],[2,3]],9]
[+,[[1,5],[2,5],[3,5]],10]
[-,[[2,1],[3,1]],1]
[+,[[2,2],[3,2],[4,2],[4,3]],13]
[*,[[3,3],[3,4],[4,4],[4,5],[5,5]],96]
[*,[[4,1],[5,1],[5,2],[5,3],[5,4]],600]
\end{lstlisting}

\section{Visualização da Solução}

De forma a representar a solução ao problema, apresentamos o tabuleiro com os números respectivos em cada posição. Ficando da seguinte maneira:

\begin{lstlisting}
Resolution:
[4,1,2,3,5]
[2,4,5,1,3]
[1,5,3,4,2]
[5,3,1,2,4]
[3,2,4,5,1]
\end{lstlisting}


\section{Visualização das Estatíscas}

Para que pudéssemos comparar a execução com as diferentes optimizações recorremos à biblioteca \textit{system} para determinar o tempo de execução e a \textit{fd\_statistics} para obter informações sobre o processamento das restrições necessárias para a resolução do problema.


\begin{lstlisting}
Statistics:
Execution time: 0 seconds
Resumptions: 3807
Entailments: 555
Prunings: 2993
Backtracks: 48
Constraints created: 62
\end{lstlisting}


\section{Resultados}

Os resultados obtidos foram bastante positivos, tanto na geração de problemas aleatórios como na resolução dos mesmos.

Conseguimos atingir um baixo tempo de execução, através da diminuição do \textit{Backtracks} executados e o menos uso possível de \textit{constraints}.

O tempo de execução varia dependendo da complexidade do problema e do tamanho do mesmo, como também poderá variar de computador para computador. Mesmo assim deixamos aqui alguns dos resultados obtidos como referência.

\subsection{Problema Exemplo nº 5}
\begin{lstlisting}
Problem:
[+,[[1,1],[1,2],[1,3]],7]
[+,[[1,4],[2,4],[2,3]],9]
[+,[[1,5],[2,5],[3,5]],10]
[-,[[2,1],[3,1]],1]
[+,[[2,2],[3,2],[4,2],[4,3]],13]
[*,[[3,3],[3,4],[4,4],[4,5],[5,5]],96]
[*,[[4,1],[5,1],[5,2],[5,3],[5,4]],600]

Resolution:
[4,1,2,3,5]
[2,4,5,1,3]
[1,5,3,4,2]
[5,3,1,2,4]
[3,2,4,5,1]

Statistics:
Execution time: 0 seconds
Resumptions: 3807
Entailments: 555
Prunings: 2993
Backtracks: 48
Constraints created: 62
\end{lstlisting}

\subsection{Problema Gerado com tamanho 7}

\begin{lstlisting}
Problem:
[+,[[1,1],[2,1],[3,1],[4,1]],13]
[+,[[1,2],[2,2],[3,2],[3,3],[2,3]],14]
[*,[[1,3],[1,4],[1,5],[2,5],[2,6]],540]
[+,[[1,6],[1,7],[2,7],[3,7],[4,7],[5,7],[5,6]],30]
[*,[[2,4],[3,4],[3,5],[4,5],[4,6],[3,6]],350]
[*,[[4,2],[5,2],[5,1],[6,1],[7,1]],1575]
[+,[[4,3],[5,3],[5,4],[6,4],[6,3],[7,3]],28]
[+,[[4,4]],6]
[*,[[5,5],[6,5],[7,5],[7,6],[6,6],[6,7],[7,7]],2880]
[*,[[6,2],[7,2]],42]
[+,[[7,4]],4]

Resolution:
[1,4,2,3,5,7,6]
[2,1,4,5,3,6,7]
[6,2,3,1,7,5,4]
[4,5,7,6,1,2,3]
[5,3,6,7,4,1,2]
[3,7,1,2,6,4,5]
[7,6,5,4,2,3,1]

Statistics:
Execution time: 0 seconds
Resumptions: 55348
Entailments: 5992
Prunings: 49973
Backtracks: 715
Constraints created: 652
\end{lstlisting}

\subsection{Problema Exemplo nº 8}

\begin{lstlisting}
Problem:
[+,[[1,1],[2,1],[2,2],[1,2],[1,3],[2,3],[2,4],[3,4]],19]
[*,[[1,4],[1,5],[2,5],[2,6],[2,7],[2,8],[3,8],[3,7]],645120]
[+,[[1,6],[1,7]],13]
[*,[[1,8]],8]
[*,[[3,1],[4,1],[5,1],[5,2],[5,3],[5,4],[4,4],[4,3]],241920]
[+,[[3,2],[3,3]],6]
[*,[[3,5],[4,5],[4,6],[5,6],[6,6],[6,7],[7,7],[7,8]],40320]
[*,[[3,6]],8]
[+,[[4,2]],3]
[*,[[4,7],[4,8],[5,8],[5,7]],70]
[+,[[5,5],[6,5],[6,4],[7,4],[7,3],[6,3],[6,2]],35]
[*,[[6,1],[7,1],[8,1],[8,2],[7,2]],13440]
[*,[[6,8]],2]
[-,[[7,5],[8,5]],-1]
[+,[[7,6],[8,6],[8,7],[8,8]],11]
[-,[[8,3],[8,4]],-3]

Resolution:
[1,2,3,4,5,6,7,8]
[2,1,5,3,4,7,8,6]
[3,5,1,2,7,8,6,4]
[4,3,2,6,8,1,5,7]
[7,6,8,5,3,4,2,1]
[8,4,7,1,6,5,3,2]
[5,7,6,8,1,2,4,3]
[6,8,4,7,2,3,1,5]

Statistics:
Execution time: 0 seconds
Resumptions: 2324
Entailments: 264
Prunings: 1729
Backtracks: 15
Constraints created: 140
Constraints created: 652
\end{lstlisting}

\subsection{Problema Exemplo nº 9}

\begin{lstlisting}
Problem:
[-,[[1,1],[1,2]],2]
[+,[[1,3],[2,3],[3,3]],8]
[*,[[1,4],[2,4],[2,5],[2,6],[1,6],[1,7]],3780]
[+,[[1,5]],5]
[+,[[1,8],[2,8],[3,8],[3,7],[4,7],[4,8],[5,8]],38]
[+,[[1,9],[2,9],[3,9]],23]
[*,[[2,1],[3,1],[3,2],[2,2]],24]
[+,[[2,7]],8]
[+,[[3,4],[4,4]],7]
[*,[[3,5],[4,5]],56]
[+,[[3,6],[4,6]],17]
[+,[[4,1],[4,2],[5,2],[5,1],[6,1],[6,2],[6,3],[5,3]],44]
[*,[[4,3]],6]
[*,[[4,9],[5,9],[6,9],[7,9],[7,8],[7,7],[8,7],[8,8],[8,9]],33600]
[+,[[5,4],[5,5],[6,5]],14]
[+,[[5,6],[6,6],[6,7],[5,7]],18]
[+,[[6,4],[7,4],[7,5],[7,6],[8,6],[9,6],[9,5],[9,4],[9,3]],43]
[*,[[6,8]],3]
[*,[[7,1],[8,1],[9,1],[9,2]],3456]
[+,[[7,2],[8,2],[8,3],[7,3]],25]
[-,[[8,4],[8,5]],1]
[*,[[9,7],[9,8],[9,9]],72]

Resolution:
[4,2,1,3,5,6,7,9,8]
[2,3,4,1,6,5,8,7,9]
[1,4,3,2,7,8,9,5,6]
[3,1,6,5,8,9,2,4,7]
[5,6,8,9,4,7,1,2,3]
[7,5,9,8,1,4,6,3,2]
[8,7,2,6,9,3,4,1,5]
[6,9,7,4,3,2,5,8,1]
[9,8,5,7,2,1,3,6,4]

Statistics:
Execution time: 0 seconds
Resumptions: 268155
Entailments: 23844
Prunings: 257699
Backtracks: 2938
Constraints created: 569
\end{lstlisting}


\section{Conclusões}

Com este trabalho fomos capazes de perceber o quão utéis poderão ser as restrições em Prolog e como estas poderão simplificar a resolução de alguns problemas, que noutra linguagem seria muito mais complexa.
Melhoramos substancialmente o conhecimento acerca desta linguagem, o que fez com que o nosso trabalho fosse estruturado e pensado desde o início, ao contrário do que aconteceu com o 1º trabalho, o que nos permitiu melhorar significativamente tanto a performance da aplicação como a nossa produtividade neste projeto.


Conseguimos assim criar uma solução fiável de resolução de problemas KenKen e uma solução totalmente aleatória, à prova de erros, de geração de problemas deste jogo.
Pelo que estamos bastante satisfeitos com as nossas melhorias e orgulhosos do nosso trabalho.



\begin{thebibliography}{4}

\bibitem{url} KenKen Math Puzzles, \url{http://www.kenken.com}

\end{thebibliography}



\end{document}
